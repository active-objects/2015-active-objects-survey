
\section{Generalities about actors and active objects}
\emph{We should probably start by  a section giving the common principles of actor and 
active object languages, giving also the vocabulary}

The active object model is  derived from the Actors
model \cite{DBLP:conf/birthday/AghaT04,agha97foundation,Agha86-book}. Actors and active objects share
a lot of concerns and advantages. A great part of the mechanisms
designed for one programming paradigm can be applied, almost
straightforwardly, to the other.


The principle of active objects is very simple: An object is said to be
active if it can be deployed on a remote machine. As a consequence,
every call to such an object will be a remote method invocation; we
call such an invocation a \emph{request}. An active object is thus an
object that treats the requests it receives, it is an object together
with a thread.

To decouple the invocation object from the invoked object, contrarily
to a classical remote invocation, the invoker is not blocked waiting
for the result instead a future object is created and represents the
result of the remote invocation.

Futures, first introduced in Multilisp \cite{Halstead85} and ABCL/f~\cite{ABCL1994}
are used as constructs for concurrency and data
flow synchronisation. Futures are language constructs that improve
concurrency in a natural and transparent way. A future represents a
result that has not been computed yet. When the result is available it
can be retrieved (automatically or manually), we then say that the
future is \emph{resolved}.
Frameworks that make use
of explicit constructs for creating futures include
Multilisp~\cite{Halstead85}, $\lambda$-calculus \cite{jlambda-fut06}, Creol
\cite{Elinar2006}, SafeFuture API \cite{SafeFutures05}, and ABCL/f
\cite{ABCL1994}. In contrast, futures are created implicitly in
frameworks like  ASP \cite{CHS:POPL04}, \cite{CHS-IC2008},
\cite{CH-book},
AmbientTalk~\cite{DedeckerCMDM-ecoop06},
ProActive~\cite{CDD:CMST06}.  Usually, in those object-oriented languages,
implicit creation corresponds to asynchronous method invocation. A key
benefit of the implicit creation is that no distinction is made
between synchronous and asynchronous operations in the program.
This way, when a method invocation is local, usual method invocation
is performed, whether when the accessed object is remote, a future is
immediately obtained.

Additionally, the futures can be accessed explicitly or implicitly. In
case of explicit access, operations like \emph{claim} and \emph{get},
\emph{touch} are used to access the future \cite{Elinar2006,ABCL1994}.
For implicit access, operations that need the real value of an object
(\emph{blocking} operations) automatically trigger synchronisation
with the future update operation. We say that futures are \emph{first
	class} if future references can be transmitted between remote
entities without requiring the future to be resolved.



\section{Criteria for comparison}
\emph{Here we suggest to summarise the points of comparison between different active 
object languages}


The principle of active objects is to associate a thread to an object, or to a set of
objects. We call \emph{activity} this notion: a thread and the objects
managed by this thread.  Two objects in different activities 
communicate by remote method invocation: when an object invokes a
method on an object in another activity, this creates a
\emph{request}; the invoker continues its execution while the invoked
object will serve the request in an asynchronous manner. Requests wait for execution in a \emph{request queue}. Like in any
object-oriented language, method invocations can return results. In
order to allow the invoker to continue its execution, a placeholder
for the expected result must be created. \emph{Futures} play this
role: a future is an empty object that will later be filled by the
result of a request. We say that a future is \emph{resolved} when its
value is known.
Overall, the active object paradigm makes programming of distributed
applications more natural, especially when applications are made of
computational entities that may have a state like objects, and that
execute in a decoupled manner because they are distributed.
While all active object models rely on the previously introduced notions, their
detailed characteristics vary in practice, which we present below:

\subsubsection{How are objects associated to activities?}\label{sec:activity}
We distinguish three different ways to map objects to threads/activities: 
\begin{itemize}
	\item \textit{Uniform Object Model.}  All objects are active objects
	with their own request queue and their own private execution thread.
	In consequence, all communications between objects occur by posting
	a request. This model is simple to formalise but its implementation
	is more difficult as it must rely on the creation of logical thread
	to allow scalability.
	\item \textit{Non Uniform Object Model.}  Some of the objects are not
	active, in which case they are only accessible by a single active
	object, they are part of its state. 
	An activity contains one active and several passive objects. Non-uniform active object models
	are much more efficient as they require less communications and
	less concurrent threads than models where each object would be
	active. Reducing the number of activities also reduces the number of
	references that are globally accessible in the system, and thus enables the
	instantiation of a large number of objects.
	\item \textit{Object Group Model.}
	In this model, an activity is made of a set of objects sharing a
	request queue and an execution thread, but all
	objects can be invoked from another activity. This
	approach improves scalability as it reduces the number of threads, but
	it is still difficult to create a lot of objects as all of them must
	be registered so that they are accessible from any other activity.
\end{itemize}

\subsubsection{How are requests scheduled?}
We distinguish three  threading models in active object languages.
\begin{description}
	\item \textit{Single-threaded.} 
	Within an active object, requests are executed sequentially without
	interleaving possibilities. 
	\item \textit{Cooperative scheduling.}
	A running request can explicitly release the execution thread to let
	another request run. Requests are not processed in parallel but they
	might interleave. Data-races are avoided.
	\item \textit{Multi-threaded.}
	Within an active object, requests are executed in parallel using
	different threads, without pausing nor yielding.
	Some data-races inside the activity is possible. This is
	called a \emph{multiactive object} model.
\end{description}

\subsubsection{Is the programmer aware of distributed aspects?}
Some active object languages use a specific syntax for asynchronous
method calls: a different operator is used to distinguish them from
synchronous method calls. This makes the programmer aware of the
places where futures are created. Generally, when asynchronous
invocation is explicit, there exists a special type for future objects.
Additionally, the futures, when they are statically identified, can be
accessed explicitly or implicitly. In case of explicit access,
operations like \emph{claim}, \emph{get} and \emph{touch} are used to
access the future. This is particularly convenient for releasing the
current thread if a future is not available in case of cooperative
scheduling (\emph{await} statement). For implicit access, operations that need the real
value of an object (\emph{blocking} operations) automatically trigger
synchronisation with the future update operation.
Implicit future creation allows transparency of distributed aspects:
there is almost no difference between a distributed program and usual
objects, whereas explicit manipulation of futures allows the
programmer to better control its execution, but requires from him a
better expertise. We say that futures
are \emph{first class} if future references can be transmitted between
remote entities without requiring the future to be resolved.





\section{Active object and Actor languages}\label{sec:overview}
\emph{Here we suggest to have a list of active object languages referring to the 
preceding classification}

Several programming languages or models rely on some form of active
objects, we review the main ones below. Especially focusing on
languages which have a formalised semantics. But before focusing on
active objects, let us first review the main works on formalisation of
futures in the context of functional languages.
Indeed, Futures
were first introduced in Multilisp \cite{Halstead85}, and thus lead to
a lot of work on their formalisation in the context of 
functional languages. We will then review main active object languages
and finish by a couple of languages that are not pure active objects
but also strongly relate the notions of distribution and concurrency to
the notion of objects.



\subsection{Futures in functional languages}
To our knowledge, the first work on formalisation by semantic rules of
Futures appeared in \cite{FlanaganFelleisen99,FlanaganFelleisen1995} and was intended at
program optimisation. This work focuses on the futures  of MultiLisp,
that are explicitly created. The authors ``compile'' a program with
futures into a low-level program that does explicit touch operations
for resolving the future, and then optimise the number of necessary touch operations.

In a similar vein,
$\lambda(fut)$ is a concurrent lambda calculus with futures.
It features non determinism primitives (cells and handles). In
\cite{jlambda-fut06}, the authors define a semantics for this calculus,
and two type systems. They use
futures with explicit creation point in the context of $\lambda$-calculus; much in
the same spirit as in Multilisp. Alice~\cite{Niehren:Sabel:Schmidt-Schauss:Schwinghammer:06} is an
ML-like language that can be considered as an implementation of
$\lambda(fut)$.



\subsection{Asynchronous Sequential Processes}
The ASP calculus we have defined~\cite{CH-book} is a distributed active object calculus with futures;  the
ProActive library \cite{CDD:CMST06} can be considered as its reference
implementation.  The ASP calculus formalises the
following characteristics of active objects:
\begin{itemize}
	\item \emph{asynchronous communications}, by a request-reply mechanism;
	\item \emph{futures}; in ASP, futures are transparent
	objects: their creation and access is implicit, futures are also
	\emph{first-class}: they can replace
	transparently any other objects and can be communicated as 
	result or parameter of remote method invocations;
	\item \emph{sequential execution} within each process, each object
	is manipulated by a single thread;
	\item \emph{imperative objects}, i.e. each object has a state and
	there is an operation for updating it.
\end{itemize}

ASP's active objects are quite similar to actors and ensure the
absence of sharing: objects live in disjoint activities. An activity
is a set of objects managed by a unique process and a unique
active object.  Active objects are accessible through global/distant
references.  They communicate through asynchronous method calls with
futures. 
ASP is a non-uniform active object model: some of the objects are
not active, in which case they are only accessible by a single active
object, they are part of its state. Non-uniform active object models
are much more efficient as they require much less communications and
much less concurrent threads than models where each object would be
active.

Our main result consists in a confluence property and its
application to the identification of a set of programs behaving
deterministically.  This property can be summarised as follows:
\begin{itemize}
	\item 
	Future updates can occur at any time without any consequence on the
	result of the computation,
	\item Execution is only characterised by the order of requests, and
	even more precisely by the order of request senders: To characterise
	uniquely an execution,
	it is sufficient to consider, for each activity, the ordered list
	of identifiers of the activities that have sent a request to this activity.
	\item Consequently, programs communicating over trees are deterministic.
\end{itemize}


The code snippet shown in Figure~\ref{fig:pasniplet} gives a typical
example of a simple ProActive piece of code. It
creates a new active object of type \code{A} on the JVM identified
by \code{Node1}. All calls to that remote object will be
asynchronous, and the access to the result might be subject to \emph{wait-by-necessity}.
% \begin{small}
\begin{figure*}[t]
	\begin{lstlisting}[frame=single]
	A a = (A) ProActive.newActive("A", params, Node1); // active object creation
	v = a.bar (...);   // Asynchronous call, no wait, v gets a future
	o.gee (v);         // No wait, even if o is a remote active object and v is still awaited
	...
	v.f (...);        // Wait-by-necessity: wait until v gets its value
	\end{lstlisting}  % \end{verbatim}
	\caption{Typical ProActive code}
	\label{fig:pasniplet}
\end{figure*}% \end{small}

\smallskip

The main advantage of ASP is that most code can be written without
caring about distribution and concurrency.  Futures are automatically
and transparently created upon method invocation on an active object.
Synchronisation is due to wait-by-necessity that occurs
upon access to a future that has not been resolved yet.
This synchronisation is performed transparently, i.e. 
there is no  construct for explicitly waiting the result of
a request. Wait-by-necessity is always
performed at the last moment, i.e. when a value is really needed.
Futures are also transparently sent between activities. This
way simple programs can be written extremely easily and rapidly.


Recent works tend to introduce local concurrency either at the actor
level
\cite{Scholliers:2014:PAM:2563745.2564410,ref:spec-actors} or at the
active object level, in particular with multiactive objects
\cite{ref:mao}.
Multiactive objects avoid deadlocks that are likely to arise
%when doing reentrant calls 
with mono-threaded active objects, and take
advantage of multi-core architectures.  
%The principle of multiactive objects is to execute
%multiple requests of an active object in parallel, while controlling
%the concurrency.
%The multiactive object model has been implemented as an extension of ProActive. This 
%extension comes as a small metalanguage to allow 
In practice, a programmer declares which requests of an active object can safely be
executed in parallel, namely which requests are \emph{compatible}. An implementation of 
multiactive objects is available in the latest version of ProActive; 
%The
%metalanguage to do that in ProActive 
it is based on the Java annotation
mechanism:
% A programmer can use these annotations on top of a class so that all objects of this 
%class are automatically multiactive objects when instanciated with the \code{newActive} 
%ProActive primitive\footnote{if there is no annotation, then objects crerated with the 
%\code{newActive} primitive remain basic active objects}. 
%Request compatibility can be specified following three steps:
%\begin{itemize}
%\item A programmer first uses the \code{@Group} annotation on top of a class to define a 
%group of requests. A group is meant to gather requests that have the same concerns 
%and/or 
%the same compatibility requirements. 
%\item Then, a programmer uses the \code{@MemberOf} annotation on top of a method 
%definition to make this method belong to a particular group (previously defined).
%\item Finally a programmer uses the \code{@Compatible} annotation to specify which group 
%is compatible with which other group, so, in extension, to specify which requests can be 
%run in parallel safely. 
%\end{itemize}
%Below is an example of an annotated
%class \code{MyClass}:
%\ttfamily\scriptsize
\setlength\abovecaptionskip{0.25mm}
\lstset{
	morekeywords={@DefineGroups,@Compatible,@DefineRules,@Group,@MemberOf},
} 
\begin{lstlisting}
@Group(name="group1", selfCompatible=true)
@Group(name="group2", selfCompatible=false)
@Compatible({"group1", "group2"})
public class MyClass {
...
@MemberOf("group1")   public ... method1(...) { ... }
@MemberOf("group2")   public ... method2(...) { ... }
}
\end{lstlisting}
In this example, groups of requests are defined with \code{@Group} and filled with
\code{@MemberOf}. Group compatibilities are specified with
\code{@Compatible}. Parameter \code{selfCompatible} defines
whether two different requests of the same group are allowed to run in
parallel.  Upon execution, a ready request is executed if it is
compatible with requests that are executing and with older
requests in the queue to avoid starvation.  


It is also possible to set a limit on the number of threads running in
parallel~\cite{henrio:hal-00916293}.  The limit can be applied in two ways: a hard limit
restrains the overall number of threads whereas a soft limit only counts threads that are
not in wait-by-necessity.  Additionally, threads can be limited per group.

Multiactive objects enable high-level implementation of scheduling policies. We use their 
features to implement the ProActive backend for ABS.


\subsection{AmbientTalk}
In AmbientTalk \cite{DedeckerCMDM-ecoop06}, there is one major difference compared to the 
other active-object models presented above: in AmbientTalk the future access is a
non-blocking operation, it is an asynchronous call that returns
another future.  There is no wait-by-necessity upon a method call on a
future, instead the method call will be performed when the future
becomes available, in the meantime a future represents the result of
this method invocation.  This differs from the approach adopted in
other frameworks where access to a future is blocking.  This approach
avoids the possibility of a deadlock as there is no synchronisation,
but programming can become tricky.
Overall, two activities can only coordinate through  callbacks.  This
inversion of control has the advantage to avoid deadlocks but breaks the program into
independent procedures where sequences of instructions are difficult to enforce.



\subsection{Creol} Creol \cite{Johnsen2006a,Elinar2006} is an active object language
that executes several requests at the same time, with only one active at a given time.
This is some form of \emph{collaborative multi-threading} based on an \emph{await}
operation that releases the active thread so that another request can continue its
execution. Typically, one would do an \emph{await} when accessing a future so that if
the future is not yet available another thread can continue its execution. In Creol
\cite{Johnsen2006a} future creation and access is explicit, in particular a specific
syntax exists for asynchronous method invocation. Creol is a uniform active object model
where each object is an active one able to receive remote method invocations.  Creol
also ensures the absence of data races, even if request execution can be interleaved,
and the result of computation is less predictable than in pure mono-threaded active
object models.


De Boer et al.   \cite{SDE:BoerCJ07}  provided 
the semantics of an
object-oriented language based on Creol; it features
active objects, asynchronous method calls, and futures. This semantics
extends Creol in the sense that it
supports first-class futures, although the future access is  still
explicit (using \emph{get} and \emph{await}). 
In the same paper, the authors also provide a proof system for proving properties
related to concurrency. 

\smallskip


Overall, explicit future access,
explicit release points, and explicit asynchronous calls make the Creol
programming model rich, but also requires more expertise from the programmer than the 
languages featuring more transparency. 


\TODO{probably move next paragraph} The cooperative multithreading model has the
advantage of having less deadlocks than mono-threaded active objects, because in
mono-threaded models like ASP a request must be finished before addressing the next one.
Indeed, when the result of a request is necessary in order to finish another one, the
Creol programmer can release the service thread, which is impossible in mono-threaded
models. In particular, accessing futures can lead to deadlock in case of re-entrant
requests. While no data race condition is possible, interleaving of the different
request services triggered by the different release points makes the behaviour more
difficult to predict (in particular the determinism properties of ASP cannot be proven
in Creol).
 The partial solution proposed by ASP and ProActive is ``first-class
futures'': since futures are implicitly created and transparently
transmitted as method parameters and results, the deadlock only occurs
if the future is really needed.  Alternatively, Creol, ABS, and JCoBox
provide explicit futures and allow the active thread to suspend itself
until a result is returned. Multi-active objects also allow an object to serve a 
re-entrant request provided it is compatible with the other ones.

\subsection{JCoBox}
JCoBox \cite{schafer2010jcobox} is an active object programming model
implemented in a language based on Java.  It integrates asynchronous
and synchronous communications as different operators, and partitions the
object space into ``coboxes'', similar to ASP's
activities. Each cobox is responsible for controlling the local
concurrency and maintaining its invariants. 
%Like in Creol, in each cobox, a single
Similarly to Creol, in each cobox a single thread is active at a time, but this thread can be released and coboxes
support collaborative multi-threading.


In Creol \cite{Johnsen2006a} all objects are active, whereas ASP
and JCoBox are non-uniform active object models: some objects are active and are invoked by asynchronous
remote requests, other objects are passive and are accessible by a
single activity (or cobox), they are transmitted by value.  References
from a cobox to an active object of another cobox are called ``far
references''. Far references can only be used to perform asynchronous
calls (\code{reference!method()}), which return futures.  Futures are
explicitly created and explicitly accessed, just as in Creol.
\code{await} performs a cooperative wait on the future, whereas
\code{get} blocks until the value of the future is received.  In
JCoBox,  a cobox may contain multiple active objects.

Cooperative multi-threading is similar and leads to the same
advantages and drawbacks as  in Creol. 


 



\subsection{JAC}

JAC \cite{lohr2006jac}  is an extension of Java that introduces a
higher level of concurrency and separates thread synchronisation from
application logic in a declarative fashion. JAC relies on a custom
pre-compiler and declarative annotations, in the form of Javadoc
comments placed before method headers or inside the code. Objects are
annotated as \code{controlled} when their threads are managed and
synchronised according to JAC's annotations. JAC relies on
compatibility annotations stating whether two methods can be executed
at the same time; two methods should be compatible if they do not
access the same variables (or if the access to those variables has
been protected by some locking mechanism).  For example, the following
code states that \code{isEmpty} can be safely executed concurrently
with \code{lookup} and with itself:
\begin{lstlisting}
/** @compatible lookup(Object), isEmpty() */
public boolean isEmpty() {.....}
\end{lstlisting} 

This way methods that can safely be run concurrently will
automatically be. Additional annotations are given for finer grain or
easier control of concurrency and synchronisation (e.g. to wait for a
guard to be verified before executing a method).  Exhaustive case
study of annotation effect, in particular in relation with inheritance
is also described in \cite{lohr2006jac}.

JAC's \emph{async} annotation provides some form of
active object behaviour: an asynchronous method is executed
independently of others in a separate thread. The main difference with
classical active objects is that classical active objects act as a unit of
concurrency: they are manipulated with a single thread and enforce the
absence of shared memory between active objects. Stating that all
methods of a class are asynchronous and mutually exclusive would
create some form of active objects, but without the absence of shared
memory enforcement. For example, in ASP and JCoBox, non-active objects
are called passive and are deeply copied when passed between
activities in order to guarantee the absence of sharing.

\smallskip We think JAC is a well designed model for declaring simply
powerful concurrency rules, but unfortunately it is not particularly
adapted to a distributed environment. Classical active objects on the
contrary provide a better encapsulation of data and concurrency but do
not provide concurrency abstractions as powerful as JAC annotations.
This way to annotate programs for concurrency is quite similar to multi-active objects 
featured by ASP, but the fact that ASP annotations are restricted to compatibility of 
requests makes it much less expressive but much simpler to use.

% \begin{figure*}[t]
% \begin{lstlisting}[frame=single]
% public class Peer {
% ...
% /** @when !runningAdd
% */
% public JoinResponse join(Peer other) { 
%   runningJoin = true;
%   //do logic
%   runningJoin = false;
% }

% /** @compatible add(Key, Serializable), join(Peer)
%     @when !runningJoin || !inSplittingZone(k)
% */
% public void add(Key k, Serializable value) { 
%   runningAdd = true;
%   // do logic
%   runningAdd = false;
% }

% private boolean inSplittingZone(Key k){ ... }

% }
% \end{lstlisting}
%  \caption{The CAN Peer implemented in JAC \TODO{CHECK: what happens if
%    the setting of flags interleaves??? is this example even possible
%    in JAC without side effects in guards or scheduler?? hOW???}}
%  \label{jac_can_peer}
% \end{figure*}

% In  our opinion multi-active objects 
% offer a simpler annotation system and a higher synchronization logic encapsulation 
% for the  programmer. On the other hand, JAC's inheritance model is
% well designed and resembles the  way our
% annotations are inherited on from class to class.



\smallskip\noindent\textbf{ABS.}\label{abs}
ABS~\cite{ref:abs} is an object-oriented programming language based on active objects that targets
modelling of distributed applications. ABS syntax is based on classes.
ABS has an object group model
based on the notion of concurrent object
group (hereafter named COG). 
%Object fields can only be accessed within the object methods. 
Asynchronous method calls and futures are explicit as in the following example:
%, and there is not yet support for first class futures.
% The following example is an asynchronous call that returns a future of
%parametric type \code{V}:
\lstset{frame=single} 
\begin{lstlisting}
Fut<V> future = object!method();
\end{lstlisting}

Requests are scheduled in a cooperative manner once again thanks to the \code{await} keyword, used as follows:
% that can be followed by a future or a condition, as in the two following examples:
\lstset{frame=single,
	morekeywords={await}}
\begin{lstlisting}
await future?;
await a > 2 && b < 3;
\end{lstlisting}
In those examples, the execution thread is released if the future is not resolved or if the condition is not fulfilled. 
%In those cases, the execution thread is handed over to another ready request. 
%A paused request is ready to resume when the future is resolved (in \code{case ~1}) or when the condition is fulfilled (in \code{case ~2}).
ABS also features a \code{get} accessor to retrieve a future's value; it blocks the execution thread until the future
is resolved: 
\lstset{frame=single,
	morekeywords={get}}
\begin{lstlisting}
V v = future.get;
\end{lstlisting}	
%The \code{get} accessor  % (and does not release the execution thread until then).
The ABS tool
suite\footnote{http://tools.hats-project.eu/} provides a wide variety of static verification engines that help
designing safe distributed and concurrent applications. Those engines
include a deadlock analyser \cite{CGM:SoSym2014}, a resource and cost analyser for cloud environments 
\cite{SACO:TACAS14,ABSresourcescost}, and general program properties verification with the ABS-Key tool~\cite{ref:key,ref:noc-abs-key}.
The ABS tool suite also includes a frontend
compiler and several backend translators into various programming
languages (Java, Haskell, Maude). 
The Java backend for ABS translates ABS programs into Java programs but is not able to
generate code that runs in a distributed manner.
%, neither in the ABS language itself nor in any of the existing backend.



\subsection{X10} 
X10 \cite{charles2005x10} is a programming language that adopts a
fairly new model, called partitioned global address space (PGAS). In this model,
computations are performed in multiple places (possibly on various computational units)
simultaneously. Data in one place is accessible remotely, and is not movable once
created. Computations inside places are locally synchronous, but inter-place activities
are asynchronous. This decouples places and ensures global parallelism. While this model
seems fundamentally different from active objects, both can be used to express the same
kind of applications, but more importantly in X10 like in ProActive, a special care has
been put to offer to the programmer a wide set of so called \emph{technical services}.
Technical services are non-functional features that are crucial to deploy and run
large-scale applications, they typically include fault-tolerance, security, code
migration, deployment on a wide set of architectures, support for several communication
protocols, etc. 

\subsection{Encore}
To be written

\section{Comparing languages} 

\emph{Here we should speak about point-to-point comparisons and every point of
	comparison that was not mentioned in the preceding section. Translation from ABS to 
	ASP is a typical example of what we want to put in this section.}

\TODO{Perhaps also fit some usecases here especially if they have been implemented using 
different models}


\section{Lessons learned and Perspectives}
\emph{Here it would be interesting to have a higher view of lessons learned, on what AO 
language to use to do what. As a perspective perhaps give the characteristics of the 
ideal active object language (can we mix some of them to make a better one?)}